\chapter{Introduction}

Financial institutions make decisions on whether to buy or sell assets based on various reasons, including: customer requests, fundamental analysis\cite{fundamental-analysis}, technical analysis\cite{technical-analysis}, top-down investing\cite{td-investing}, bottom-up investing\cite{bu-investing} and many more. 
The high-level trading strategies oftentimes define the purpose of their business and how the institution positions itself in the various financial markets and, if existent, towards its customers. 
Regardless of the high-level trading strategy that is being applied, the invariable outcome is the decision to buy or sell assets. 
Hence, an placement strategy aims to execute orders and subsequently buy or sell assets to a favourable price.

\section{Problem Statement}

\section{Research objectives}

The main objective of this research is to develop a machine learning based model of limit-order placements in an order driven crypto currency market that can be utilised to optimize buying and selling of crypto currencies. To achieve this, we firstly conduct several experiments to analyse the behaviour of previously proposed models in a controlled environment by utilising historical order book data with the aim to identify the limitation of these models. We then propose a reinforcement learning implementation and study the results with respect to the aforementioned limitations. Questions to be answered include:

How does trading time horizon and inventory affect execution?
How does market liquidity provided by bid and ask orders affect execution?
Is the performance of reinforcement learning limited to the findings of an empirical model that models the execution probability?
Another focus of this research is the aim to derive patterns from order book data. Particularly, order book events are observed and an attempt is made to detect behavioural patterns of traders participating in the market. Hence the objective is:

How can data, which is derived from a limit order book, be used as features in a reinforcement learning environment in order to contribute to the optimization of order execution?
Lastly, we make an attempt to improve the execution strategy by incorporating the found patterns. Therefore the aim is to answer:

Does pattern recognition of order book data allow to overcome the limitations of order execution proposed by statistical models and other machine learning approaches?


\section{Contributions}

\section{Document structure}

