\chapter{Experimental Setup}
\label{chap:setup}

\section{Feature engineering}

\section{Execution Environment}

\section{Market making Environment}

\section{Q-Learning agent}

One of the reinforcement learning algorithms used in this work is known as \textit{Q-Learning}. 
The name derives from the application of the previously presented Q-function (Eq. \ref{eq:q-function}), more specifically its use within the Bellman equation (\ref{eq:bellman}) that undergoes an iterative update.
\\
\\
The general idea is ... 

\section{Deep Q-Network agent}
